\documentclass[a4paper]{article}

\usepackage[T1]{fontenc}
\usepackage[utf8]{inputenc}
\usepackage[russian]{babel}
\usepackage{amsmath}
\usepackage{amssymb}
\usepackage{amsthm}
\usepackage{graphicx}
\usepackage[colorinlistoftodos]{todonotes}
\usepackage{csquotes}
\usepackage{xstring}
\usepackage{xfrac}

%%math
%problem
\newcounter{prb}
\newcommand{\prb}[1]{\addtocounter{prb}{1}\vskip 2mm\noindent{\bf \arabic{prb}. }{\it #1\/}\vskip 2mm}
%statement
\newcommand{\stt}[1]{\vskip 1mm\noindent{\bf Утверждение. }{\it #1\/}\vskip 1mm}
%misc
\newcommand{\lleq}{\leqslant}
\newcommand{\ggeq}{\geqslant}

\newtheoremstyle{dotless}{}{}{\itshape}{}{\bfseries}{}{ }{}
\theoremstyle{dotless}
\newtheorem*{answer}{Ответ:}

\begin{document}

\section*{ЗАДАЧА 1}


\begin{answer}
$2^n$
\end{answer}
\begin{proof}
Докажем индукцией по $n$ следующее утверждение: для всех $X$ верно, что если $|X|=n$, то $|2^X|=2^n$.

{\bf База.} Для случая $n=0$ можно проверить, что у пустого множества $X$ всего одно подмножество: пустое. То есть $|2^X|=2^0$.

{\bf Индукционный переход.} Пусть для некоторого $k$ доказываемое утверждение выполняется. Докажем, что оно выполняется и для $k+1$.

Зафиксируем произвольное $X$, $|X|=k+1$, и возьмем какой-нибудь $x\in X$. Поскольку $|X\setminus \{x\}| = k$, из индукционного предположения следует, что $|2^{X\setminus \{x\}}|=2^k$. Теперь разобьем $2^X$ на парочки: элементу $A\in 2^X$ соответствует $A\cup \{x\}$, если $x\not\in A$, и $A\setminus \{x\}$ в противном случае. 

Покажем, что парочек столько же, сколько элементов в $2^{X\setminus \{x\}}$. Каждой парочке $\{A, A'\}$ поставим в соответствие $A\cap A'\in 2^{X\setminus \{x\}}$, а каждому элементу $B \in 2^{X\setminus \{x\}}$ поставим в соответствие $\{B, B\cup \{x\}\}$. Проверим, что эти соответствия --- обратные функции друг друга:  если $x\in A'$, парочка $\{A, A'\}$ перейдет в $A\in 2^{X\setminus \{x\}}$, а $A\in 2^{X\setminus \{x\}}$ в свою очередь перейдет в $\{A, A'\}$. И наоборот: $B \in 2^{X\setminus \{x\}}$ перейдет в $\{B, B\cup\{x\}\}$, а затем вернется в $B\cap (B\cup\{x\})=B\in 2^{X\setminus \{x\}}$. Поэтому у нас есть биекция между парочками элементов $2^X$ и множеством $2^{X\setminus \{x\}}$. Так что в $2^X$ в два раза больше элементов, чем в $2^{X\setminus \{x\}}$, то есть $|2^X|=2^{k+1}$.
\end{proof}

\end{document}