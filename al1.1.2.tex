\documentclass[a4paper]{article}

\usepackage[T1]{fontenc}
\usepackage[utf8]{inputenc}
\usepackage[russian]{babel}
\usepackage{amsmath}
\usepackage{amssymb}
\usepackage{amsthm}
\usepackage{graphicx}
\usepackage[colorinlistoftodos]{todonotes}
\usepackage{csquotes}
\usepackage{xstring}
\usepackage{xfrac}

%%math
%misc
\newcommand{\lleq}{\leqslant}
\newcommand{\ggeq}{\geqslant}

\newtheoremstyle{dotless}{}{}{\itshape}{}{\bfseries}{}{ }{}
\theoremstyle{dotless}
\newtheorem*{answer}{Ответ:}

\begin{document}

\section*{ЗАДАЧА 2}

\begin{answer}
$A\cap B = A \setminus (A \setminus B)$
\end{answer}

\begin{proof}
Сначала докажем, что $A\cap B \subset A \setminus (A \setminus B)$. 

Пусть $x\in A\cap B$. Tогда $x \in A$ и $x\in B$, значит $x\not\in A \setminus B$. Поскольку $x \in A$ и $x\not\in A \setminus B$, то $x\in A \setminus (A \setminus B)$.

Теперь докажем, что $A\cap B \supset A \setminus (A \setminus B)$.

Пусть $x\in A \setminus (A \setminus B)$. Тогда $x\in A$ и $x\not\in A \setminus B$. Последнее означает, что $x\in A \implies x\in B$. Поскольку $x\in A$, то $x\in B$. Поэтому $x \in A\cap B$.
\end{proof}

\begin{answer}
Разность множеств $A\setminus B$ нельзя выразить через пересечение и объединение.
\end{answer}

\begin{proof}
Рассмотрим набор множеств $M=\{\,A, B, A\cap B, A\cup B\,\}$. Можно проверить перебором, что это множество замкнуто относительно операций пересечения и объединения.

Теперь рассмотрим множество $F$ всех формальных выражений, в которые элементы $M$ входят в качестве операндов, а пересечение и объединение --- в качестве операций. Поскольку $M$ замкнуто относительно пересечения и объединения, результат подсчета любого такого формального выражения лежит в $M$.

Если бы множество $A\setminus B$ можно было выразить, используя $A$, $B$, пересечение и объединение, то существовало бы соответствующее формальное выражение $p\in F$, результат подсчета которого равнялся множеству $A\setminus B$. Но поскольку$A\setminus B \not\in M$, такого формального выражения нет, так что разность $A\setminus B$ нельзя выразить через пересечение и объединение.
\end{proof}

\end{document}