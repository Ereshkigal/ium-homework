\documentclass[a4paper]{article}

\usepackage[T1]{fontenc}
\usepackage[utf8]{inputenc}
\usepackage[russian]{babel}
\usepackage{amsmath}
\usepackage{amssymb}
\usepackage{amsthm}
\usepackage{graphicx}
\usepackage[colorinlistoftodos]{todonotes}
\usepackage{csquotes}
\usepackage{xstring}
\usepackage{xfrac}

\binoppenalty=10000
\relpenalty=10000

%%math
%problem
\newcounter{prb}
\newcommand{\prb}[1]{\addtocounter{prb}{1}\vskip 2mm\noindent{\bf \arabic{prb}. }{\it #1\/}\vskip 2mm}
%statement
\newcommand{\stt}[1]{\vskip 1mm\noindent{\bf Утверждение. }{\it #1\/}\vskip 1mm}
%misc
\newcommand{\lleq}{\leqslant}
\newcommand{\ggeq}{\geqslant}

\newtheoremstyle{dotless}{}{}{\itshape}{}{\bfseries}{}{ }{}
\theoremstyle{dotless}
\newtheorem*{answer}{Ответ:}

\begin{document}

\section*{ЗАДАЧА 3а}

\begin{answer}
$4!$ 
\end{answer}
\begin{proof}
Докажем по индукции, что $n$ элементов можно переставить $n!$ способами. 

{\bf База.} Всего существует одна перестановка нуля элементов: пустая.

{\bf Индукционный переход.} Пусть известно, что $k$ элементов можно переставить $k!$ способами. Докажем, что $k+1$ элемент можно переставить $(k+1)!$ способом.

Зафиксируем элемент с номером $l$ и рассмотрим отображение $f\colon P_{k+1}\to P_k$, из множества перестановок $k+1$ элементов в множество перестановок $k$ элементов, которое вычеркивает элемент с номером $l$. Для всех $x\in P_k$ известно, что $|f^{-1}(x)|=k+1$. Поскольку $f$ сюръективно, и $|P_k|=k!$, можно посчитать количество элементов в $P_{k+1}$: $(k+1)\,k!=(k+1)!$.
 
В слове \enquote{урок} четыре разные буквы, искомое количество слов это количество всевозможных перестановок этих четырех букв: $4!$.
\end{proof}

\end{document}